% File is screwed  -*- coding: utf-8-unix -*-

% Please fill in
% the date of your submission: DD/MM/YYYY
% the volume and number of the TAL issue you are submitting to: VV-NN
% whether this is the first submission (1) or a second submission after first review round (2): R

\submitted[DD/MM/YYYY]{TAL~VV-NN}{R}

% Two commands to be used only for the final version of accepted papers
% instead of the \submitted one
% \setcounter{page}{}
% Put the first page number between {}
% \journal{TAL. Volume VOL -- n°X/YYYY}{F}{L}
% VOL: TAL volume; X:  TAL number in the volume; YYYY: year; 
% F: first page number of the paper; L: last page number

\author{Firstname$_1$ Lastname$_1$\fup{*} \andauthor Firstname$_2$ Lastname$_2$\fup{**}  } 

\address{%
\fup{*} First affiliation\\
\fup{**} Second affiliation
}

\abstract{ In this paper, we study the formalisation of a dialogue
management system using proof-search on top of a linear logic.

We argue that linear logic is the natural formalism to implement
information-state dialog management.  We give particular attention to
modelling question-answering sequences, including clarification
requests, and argue that metavariables, arising from unification in
the proof search, play a decisive role in providing a natural
formalisation.

We show that our framework is not only well suited from a theoretical
perspective, but it is also suitable for implementation, having
implemented it and as well as several examples on top of it.  }

\keywords{
Symbolic dialogue management,
Linear logic,
metavariables,
question answering,
clarification requests
}

\motscles{
mot-cl\'e$_1$,
mot-cl\'e$_2$,
mot-cl\'e$_3$.
}

\resume{
Le r\'esum\'e est \`a \'ecrire ici.
}
