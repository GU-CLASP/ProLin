% File is screwed  -*- coding: utf-8-unix -*-

% Please fill in
% the date of your submission: DD/MM/YYYY
% the volume and number of the TAL issue you are submitting to: VV-NN
% whether this is the first submission (1) or a second submission after first review round (2): R

% \submitted[DD/MM/YYYY]{TAL~VV-NN}{R}

% Two commands to be used only for the final version of accepted papers
% instead of the \submitted one
\setcounter{page}{1}
% Put the first page number between {}
\journal{TAL. Volume 61 -- n°3/2020}{43}{67}
% VOL: TAL volume; X:  TAL number in the volume; YYYY: year; 
% F: first page number of the paper; L: last page number

\author{Vladislav Maraev\fup{*}\andauthor Jean-Philippe Bernardy\fup{*}  \andauthor Jonathan~Ginzburg\fup{**}} 

\address{%
\fup{*} Centre for Linguistic Theory and Studies in Probability (CLASP), Department of Philosophy, Linguistics and Theory of Science, University of~Gothenburg  \\
\fup{**} Laboratoire de Linguistique Formelle (LLF), CNRS – UMR 7110, Université de~Paris  
}

\abstract{ In this paper, we study the formalisation of a dialogue
management system using proof-search on top of a linear logic.
%
We argue that linear logic is the natural formalism to implement
information-state dialogue management.  We give particular attention to
modelling question-answering sequences, including clarification
requests, and argue that metavariables, arising from unification in
the proof search, play a decisive role in providing a natural
formalisation.
%
We show that our framework is not only well suited from a theoretical
perspective, but it is also suitable for implementation which we
exemplify with a small scale implementation.  }

\keywords{
Symbolic dialogue management,
Linear logic,
Metavariables,
Question answering,
Clarification requests
}

\motscles{
Gestion de dialogue symbolique,
Logique linéaire,
Métavariables,\\
Question/Réponse,
Demande de clarification
}

\resume{ Cet article propose une formalisation de la gestion de
dialogue via la recherche de preuves de formules de la logique
linéaire.
%
C'est-à-dire que nous proposons que la logique linéaire constitue une
base naturelle de la formalisation de systèmes de gestion de dialogue
basé sur un état d'information.  Nous prêtons une attention
particulière à la modelisation des séquences de questions-reponses (y
compris les demandes de clarification), et nous arguons que les
métavariables, résultant des unifications issue de la recherche de
preuves, jouent un rôle décisif dans la formalisation.
%
Nous montrons que notre système est non seulement adéquat d'un point
de vue théorique, mais également d'un point de vue pratique. Ainsi,
nous complétons notre argument d'une implementation d'un système de
recherche de preuve générique, ainsi que d'un exemple de gestion de
dialogue l'utilisant.  }
