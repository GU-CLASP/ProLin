% Created 2021-08-19 Thu 21:12
% Intended LaTeX compiler: pdflatex
\documentclass[11pt]{article}
\PassOptionsToPackage{hidelinks}{hyperref}
\usepackage[utf8]{inputenc}
\usepackage[T1]{fontenc}
\usepackage{graphicx}
\usepackage{grffile}
\usepackage{longtable}
\usepackage{wrapfig}
\usepackage{rotating}
\usepackage[normalem]{ulem}
\usepackage{amsmath}
\usepackage{textcomp}
\usepackage{amssymb}
\usepackage{capt-of}
\usepackage{hyperref}
\usepackage[small]{caption}
\pdfpagewidth=8.5in
\pdfpageheight=11in
\usepackage{ijcai21}
%% ODER: format ==         = "\mathrel{==}"
%% ODER: format /=         = "\neq "
%
%
\makeatletter
\@ifundefined{lhs2tex.lhs2tex.sty.read}%
  {\@namedef{lhs2tex.lhs2tex.sty.read}{}%
   \newcommand\SkipToFmtEnd{}%
   \newcommand\EndFmtInput{}%
   \long\def\SkipToFmtEnd#1\EndFmtInput{}%
  }\SkipToFmtEnd

\newcommand\ReadOnlyOnce[1]{\@ifundefined{#1}{\@namedef{#1}{}}\SkipToFmtEnd}
\usepackage{amstext}
\usepackage{amssymb}
\usepackage{stmaryrd}
\DeclareFontFamily{OT1}{cmtex}{}
\DeclareFontShape{OT1}{cmtex}{m}{n}
  {<5><6><7><8>cmtex8
   <9>cmtex9
   <10><10.95><12><14.4><17.28><20.74><24.88>cmtex10}{}
\DeclareFontShape{OT1}{cmtex}{m}{it}
  {<-> ssub * cmtt/m/it}{}
\newcommand{\texfamily}{\fontfamily{cmtex}\selectfont}
\DeclareFontShape{OT1}{cmtt}{bx}{n}
  {<5><6><7><8>cmtt8
   <9>cmbtt9
   <10><10.95><12><14.4><17.28><20.74><24.88>cmbtt10}{}
\DeclareFontShape{OT1}{cmtex}{bx}{n}
  {<-> ssub * cmtt/bx/n}{}
\newcommand{\tex}[1]{\text{\texfamily#1}}	% NEU

\newcommand{\Sp}{\hskip.33334em\relax}


\newcommand{\Conid}[1]{\mathit{#1}}
\newcommand{\Varid}[1]{\mathit{#1}}
\newcommand{\anonymous}{\kern0.06em \vbox{\hrule\@width.5em}}
\newcommand{\plus}{\mathbin{+\!\!\!+}}
\newcommand{\bind}{\mathbin{>\!\!\!>\mkern-6.7mu=}}
\newcommand{\rbind}{\mathbin{=\mkern-6.7mu<\!\!\!<}}% suggested by Neil Mitchell
\newcommand{\sequ}{\mathbin{>\!\!\!>}}
\renewcommand{\leq}{\leqslant}
\renewcommand{\geq}{\geqslant}
\usepackage{polytable}

%mathindent has to be defined
\@ifundefined{mathindent}%
  {\newdimen\mathindent\mathindent\leftmargini}%
  {}%

\def\resethooks{%
  \global\let\SaveRestoreHook\empty
  \global\let\ColumnHook\empty}
\newcommand*{\savecolumns}[1][default]%
  {\g@addto@macro\SaveRestoreHook{\savecolumns[#1]}}
\newcommand*{\restorecolumns}[1][default]%
  {\g@addto@macro\SaveRestoreHook{\restorecolumns[#1]}}
\newcommand*{\aligncolumn}[2]%
  {\g@addto@macro\ColumnHook{\column{#1}{#2}}}

\resethooks

\newcommand{\onelinecommentchars}{\quad-{}- }
\newcommand{\commentbeginchars}{\enskip\{-}
\newcommand{\commentendchars}{-\}\enskip}

\newcommand{\visiblecomments}{%
  \let\onelinecomment=\onelinecommentchars
  \let\commentbegin=\commentbeginchars
  \let\commentend=\commentendchars}

\newcommand{\invisiblecomments}{%
  \let\onelinecomment=\empty
  \let\commentbegin=\empty
  \let\commentend=\empty}

\visiblecomments

\newlength{\blanklineskip}
\setlength{\blanklineskip}{0.66084ex}

\newcommand{\hsindent}[1]{\quad}% default is fixed indentation
\let\hspre\empty
\let\hspost\empty
\newcommand{\NB}{\textbf{NB}}
\newcommand{\Todo}[1]{$\langle$\textbf{To do:}~#1$\rangle$}

\EndFmtInput
\makeatother
%
%
%
%
%
%
% This package provides two environments suitable to take the place
% of hscode, called "plainhscode" and "arrayhscode". 
%
% The plain environment surrounds each code block by vertical space,
% and it uses \abovedisplayskip and \belowdisplayskip to get spacing
% similar to formulas. Note that if these dimensions are changed,
% the spacing around displayed math formulas changes as well.
% All code is indented using \leftskip.
%
% Changed 19.08.2004 to reflect changes in colorcode. Should work with
% CodeGroup.sty.
%
\ReadOnlyOnce{polycode.fmt}%
\makeatletter

\newcommand{\hsnewpar}[1]%
  {{\parskip=0pt\parindent=0pt\par\vskip #1\noindent}}

% can be used, for instance, to redefine the code size, by setting the
% command to \small or something alike
\newcommand{\hscodestyle}{}

% The command \sethscode can be used to switch the code formatting
% behaviour by mapping the hscode environment in the subst directive
% to a new LaTeX environment.

\newcommand{\sethscode}[1]%
  {\expandafter\let\expandafter\hscode\csname #1\endcsname
   \expandafter\let\expandafter\endhscode\csname end#1\endcsname}

% "compatibility" mode restores the non-polycode.fmt layout.

\newenvironment{compathscode}%
  {\par\noindent
   \advance\leftskip\mathindent
   \hscodestyle
   \let\\=\@normalcr
   \let\hspre\(\let\hspost\)%
   \pboxed}%
  {\endpboxed\)%
   \par\noindent
   \ignorespacesafterend}

\newcommand{\compaths}{\sethscode{compathscode}}

% "plain" mode is the proposed default.
% It should now work with \centering.
% This required some changes. The old version
% is still available for reference as oldplainhscode.

\newenvironment{plainhscode}%
  {\hsnewpar\abovedisplayskip
   \advance\leftskip\mathindent
   \hscodestyle
   \let\hspre\(\let\hspost\)%
   \pboxed}%
  {\endpboxed%
   \hsnewpar\belowdisplayskip
   \ignorespacesafterend}

\newenvironment{oldplainhscode}%
  {\hsnewpar\abovedisplayskip
   \advance\leftskip\mathindent
   \hscodestyle
   \let\\=\@normalcr
   \(\pboxed}%
  {\endpboxed\)%
   \hsnewpar\belowdisplayskip
   \ignorespacesafterend}

% Here, we make plainhscode the default environment.

\newcommand{\plainhs}{\sethscode{plainhscode}}
\newcommand{\oldplainhs}{\sethscode{oldplainhscode}}
\plainhs

% The arrayhscode is like plain, but makes use of polytable's
% parray environment which disallows page breaks in code blocks.

\newenvironment{arrayhscode}%
  {\hsnewpar\abovedisplayskip
   \advance\leftskip\mathindent
   \hscodestyle
   \let\\=\@normalcr
   \(\parray}%
  {\endparray\)%
   \hsnewpar\belowdisplayskip
   \ignorespacesafterend}

\newcommand{\arrayhs}{\sethscode{arrayhscode}}

% The mathhscode environment also makes use of polytable's parray 
% environment. It is supposed to be used only inside math mode 
% (I used it to typeset the type rules in my thesis).

\newenvironment{mathhscode}%
  {\parray}{\endparray}

\newcommand{\mathhs}{\sethscode{mathhscode}}

% texths is similar to mathhs, but works in text mode.

\newenvironment{texthscode}%
  {\(\parray}{\endparray\)}

\newcommand{\texths}{\sethscode{texthscode}}

% The framed environment places code in a framed box.

\def\codeframewidth{\arrayrulewidth}
\RequirePackage{calc}

\newenvironment{framedhscode}%
  {\parskip=\abovedisplayskip\par\noindent
   \hscodestyle
   \arrayrulewidth=\codeframewidth
   \tabular{@{}|p{\linewidth-2\arraycolsep-2\arrayrulewidth-2pt}|@{}}%
   \hline\framedhslinecorrect\\{-1.5ex}%
   \let\endoflinesave=\\
   \let\\=\@normalcr
   \(\pboxed}%
  {\endpboxed\)%
   \framedhslinecorrect\endoflinesave{.5ex}\hline
   \endtabular
   \parskip=\belowdisplayskip\par\noindent
   \ignorespacesafterend}

\newcommand{\framedhslinecorrect}[2]%
  {#1[#2]}

\newcommand{\framedhs}{\sethscode{framedhscode}}

% The inlinehscode environment is an experimental environment
% that can be used to typeset displayed code inline.

\newenvironment{inlinehscode}%
  {\(\def\column##1##2{}%
   \let\>\undefined\let\<\undefined\let\\\undefined
   \newcommand\>[1][]{}\newcommand\<[1][]{}\newcommand\\[1][]{}%
   \def\fromto##1##2##3{##3}%
   \def\nextline{}}{\) }%

\newcommand{\inlinehs}{\sethscode{inlinehscode}}

% The joincode environment is a separate environment that
% can be used to surround and thereby connect multiple code
% blocks.

\newenvironment{joincode}%
  {\let\orighscode=\hscode
   \let\origendhscode=\endhscode
   \def\endhscode{\def\hscode{\endgroup\def\@currenvir{hscode}\\}\begingroup}
   %\let\SaveRestoreHook=\empty
   %\let\ColumnHook=\empty
   %\let\resethooks=\empty
   \orighscode\def\hscode{\endgroup\def\@currenvir{hscode}}}%
  {\origendhscode
   \global\let\hscode=\orighscode
   \global\let\endhscode=\origendhscode}%

\makeatother
\EndFmtInput
%
\usepackage{soul}
\usepackage{url}
\usepackage{times}
\renewcommand*\ttdefault{txtt}
\usepackage{graphicx}
\urlstyle{same}
\usepackage{newunicodechar}
\newunicodechar{ }{{\,}}
\newunicodechar{¬}{\ensuremath{\neg}} % 
\newunicodechar{²}{^2}
\newunicodechar{³}{^3}
\newunicodechar{·}{\ensuremath{\cdot}}
\newunicodechar{¹}{^1}
\newunicodechar{×}{\ensuremath{\times}} % 
\newunicodechar{÷}{\ensuremath{\div}} % 
\newunicodechar{ʳ}{^r}
\newunicodechar{ˡ}{^l}
\newunicodechar{Γ}{\ensuremath{\Gamma}}   %
\newunicodechar{Δ}{\ensuremath{\Delta}} % 
\newunicodechar{Η}{\ensuremath{\textrm{H}}} % 
\newunicodechar{Θ}{\ensuremath{\Theta}} % 
\newunicodechar{Λ}{\ensuremath{\Lambda}} % 
\newunicodechar{Ξ}{\ensuremath{\Xi}} % 
\newunicodechar{Π}{\ensuremath{\Pi}}   %
\newunicodechar{Σ}{\ensuremath{\Sigma}} % 
\newunicodechar{Φ}{\ensuremath{\Phi}} % 
\newunicodechar{Ψ}{\ensuremath{\Psi}} % 
\newunicodechar{Ω}{\ensuremath{\Omega}} % 
\newunicodechar{α}{\ensuremath{\mathnormal{\alpha}}}
\newunicodechar{α}{\ensuremath{\mathnormal{\alpha}}} % 
\newunicodechar{β}{\ensuremath{\beta}} % 
\newunicodechar{γ}{\ensuremath{\mathnormal{\gamma}}} % 
\newunicodechar{δ}{\ensuremath{\mathnormal{\delta}}} % 
\newunicodechar{ε}{\ensuremath{\mathnormal{\varepsilon}}} % 
\newunicodechar{ζ}{\ensuremath{\mathnormal{\zeta}}} % 
\newunicodechar{η}{\ensuremath{\mathnormal{\eta}}} % 
\newunicodechar{θ}{\ensuremath{\mathnormal{\theta}}} % 
\newunicodechar{ι}{\ensuremath{\mathnormal{\iota}}} % 
\newunicodechar{κ}{\ensuremath{\mathnormal{\kappa}}} % 
\newunicodechar{λ}{\ensuremath{\mathnormal{\lambda}}} % 
\newunicodechar{μ}{\ensuremath{\mathnormal{\mu}}} % 
\newunicodechar{ν}{\ensuremath{\mathnormal{\mu}}} % 
\newunicodechar{ξ}{\ensuremath{\mathnormal{\xi}}} % 
\newunicodechar{π}{\ensuremath{\mathnormal{\pi}}}
\newunicodechar{π}{\ensuremath{\mathnormal{\pi}}} % 
\newunicodechar{ρ}{\ensuremath{\mathnormal{\rho}}} % 
\newunicodechar{σ}{\ensuremath{\mathnormal{\sigma}}} % 
\newunicodechar{τ}{\ensuremath{\mathnormal{\tau}}} % 
\newunicodechar{φ}{\ensuremath{\mathnormal{\varphi}}} % 
\newunicodechar{χ}{\ensuremath{\mathnormal{\chi}}} % 
\newunicodechar{ψ}{\ensuremath{\mathnormal{\psi}}} % 
\newunicodechar{ω}{\ensuremath{\mathnormal{\omega}}} %  
\newunicodechar{ϕ}{\ensuremath{\mathnormal{\phi}}} % 
\newunicodechar{ϵ}{\ensuremath{\mathnormal{\epsilon}}} % 
\newunicodechar{ᵏ}{^k}
\newunicodechar{ᵢ}{_i} % 
\newunicodechar{ }{{\,}}
\newunicodechar{′}{\ensuremath{^\prime}}  % 
\newunicodechar{″}{\ensuremath{^\second}} % 
\newunicodechar{‴}{\ensuremath{^\third}}  % 
\newunicodechar{ⁱ}{^i}
\newunicodechar{⁵}{\ensuremath{^5}}
\newunicodechar{⁺}{\ensuremath{^+}} %% 
\newunicodechar{⁺}{^+}
\newunicodechar{⁻}{\ensuremath{^-}} %% 
\newunicodechar{ⁿ}{^n}
\newunicodechar{₀}{\ensuremath{_0}} % 
\newunicodechar{₁}{\ensuremath{_1}}
\newunicodechar{₁}{_1}
\newunicodechar{₂}{\ensuremath{_2}}
\newunicodechar{₂}{_2}
\newunicodechar{₊}{\ensuremath{_+}} %% 
\newunicodechar{₋}{\ensuremath{_-}} %% 
\newunicodechar{ℂ}{\ensuremath{\mathbb{C}}} % 
\newunicodechar{ℕ}{\mathbb{N}} % 
\newunicodechar{ℝ}{\ensuremath{\mathbb{R}}} % 
\newunicodechar{ℤ}{\ensuremath{\mathbb{Z}}} % 
\newunicodechar{ℳ}{\mathcal{M}}
\newunicodechar{⅋}{\ensuremath{\parr}} % 
\newunicodechar{←}{\ensuremath{\leftarrow}} % 
\newunicodechar{↑}{\ensuremath{\uparrow}} % 
\newunicodechar{→}{\ensuremath{\rightarrow}} %
\newunicodechar{↔}{\ensuremath{\leftrightarrow}} % 
\newunicodechar{↖}{\nwarrow} %
\newunicodechar{↗}{\nearrow} %
\newunicodechar{↝}{\ensuremath{\leadsto}}
\newunicodechar{↦}{\ensuremath{\mapsto}}
\newunicodechar{⇆}{\ensuremath{\leftrightarrows}} % 
\newunicodechar{⇐}{\ensuremath{\Leftarrow}} % 
\newunicodechar{⇒}{\ensuremath{\Rightarrow}} % 
\newunicodechar{⇔}{\ensuremath{\Leftrightarrow}} % 
\newunicodechar{∀}{\ensuremath{\forall}}   %
\newunicodechar{∃}{\ensuremath{\exists}} % 
\newunicodechar{∅}{\ensuremath{\varnothing}} % 
\newunicodechar{∈}{\ensuremath{\in}}
\newunicodechar{∉}{\ensuremath{\not\in}} % 
\newunicodechar{∋}{\ensuremath{\ni}}  % 
\newunicodechar{∎}{\ensuremath{\qed}}
\newunicodechar{∏}{\prod}
\newunicodechar{∑}{\sum}
\newunicodechar{∗}{\ensuremath{\ast}} % 
\newunicodechar{∘}{\ensuremath{\circ}} % 
\newunicodechar{∙}{\ensuremath{\bullet}} % 
\newunicodechar{∙}{\ensuremath{\cdot}}
\newunicodechar{∞}{\ensuremath{\infty}} % 
\newunicodechar{∣}{\ensuremath{\mid}} % 
\newunicodechar{∧}{\wedge}% 
\newunicodechar{∨}{\vee}% 
\newunicodechar{∩}{\ensuremath{\cap}} % 
\newunicodechar{∪}{\ensuremath{\cup}} % 
\newunicodechar{∫}{\int}
\newunicodechar{∷}{::} % 
\newunicodechar{∼}{\ensuremath{\sim}} % 
\newunicodechar{≃}{\ensuremath{\simeq}} % 
\newunicodechar{≅}{\ensuremath{\cong}} % 
\newunicodechar{≈}{\ensuremath{\approx}} % 
\newunicodechar{≜}{\ensuremath{\stackrel{\scriptscriptstyle {\triangle}}{=}}} 
\newunicodechar{≜}{\triangleq}
\newunicodechar{≝}{\ensuremath{\stackrel{\scriptscriptstyle {\text{def}}}{=}}}
\newunicodechar{≟}{\ensuremath{\stackrel {_\text{\textbf{?}}}{\text{\textbf{=}}\negthickspace\negthickspace\text{\textbf{=}}}}}
\newunicodechar{≠}{\ensuremath{\neq}}% 
\newunicodechar{≡}{\ensuremath{\equiv}}% 
\newunicodechar{≤}{\ensuremath{\le}} % 
\newunicodechar{≥}{\ensuremath{\ge}} % 
\newunicodechar{⊂}{\ensuremath{\subset}} % 
\newunicodechar{⊃}{\ensuremath{\supset}} %  
\newunicodechar{⊆}{\ensuremath{\subseteq}} %  
\newunicodechar{⊇}{\ensuremath{\supseteq}} % 
\newunicodechar{⊎}{\ensuremath{\uplus}} %
\newunicodechar{⊑}{\ensuremath{\sqsubseteq}} % 
\newunicodechar{⊒}{\ensuremath{\sqsupseteq}} % 
\newunicodechar{⊓}{\ensuremath{\sqcap}} % 
\newunicodechar{⊔}{\ensuremath{\sqcup}} % 
\newunicodechar{⊕}{\ensuremath{\oplus}} % 
\newunicodechar{⊗}{\ensuremath{\otimes}} % 
\newunicodechar{⊢}{\ensuremath{\vdash}} % 
\newunicodechar{⊥}{\ensuremath{\bot}} % bottom
\newunicodechar{⊧}{\models} % 
\newunicodechar{⊨}{\models} % 
\newunicodechar{⊩}{\Vdash}
\newunicodechar{⊸}{\ensuremath{\multimap}} % 
\newunicodechar{⋁}{\ensuremath{\bigvee}}
\newunicodechar{⋃}{\ensuremath{\bigcup}} % 
\newunicodechar{⋄}{\ensuremath{\diamond}} % 
\newunicodechar{⋆}{\ensuremath{\star}} % 
\newunicodechar{⋮}{\ensuremath{\vdots}} % 
\newunicodechar{⋯}{\ensuremath{\cdots}} % 
\newunicodechar{─}{---}
\newunicodechar{□}{\ensuremath{\square}} % 
\newunicodechar{▹}{\ensuremath{\rhd}} % 
\newunicodechar{◃}{\triangleleft{}}
\newunicodechar{◇}{\ensuremath{\diamond}} % 
\newunicodechar{◽}{\ensuremath{\square}}
\newunicodechar{★}{\ensuremath{\star}}   %
\newunicodechar{♭}{\ensuremath{\flat}} % 
\newunicodechar{♯}{\ensuremath{\sharp}} % 
\newunicodechar{✓}{\ensuremath{\checkmark}} % 
\newunicodechar{⟂}{\ensuremath{^\bot}} % PERPENDICULAR 
\newunicodechar{⟦}{\ensuremath{\llbracket}}
\newunicodechar{⟧}{\ensuremath{\rrbracket}}
\newunicodechar{⟨}{\ensuremath{\langle}} % 
\newunicodechar{⟩}{\ensuremath{\rangle}} % 
\newunicodechar{⟶}{{\longrightarrow}}
\newunicodechar{⟹}{\ensuremath{\Longrightarrow}} % 
\newunicodechar{𝒟}{\ensuremath{\mathcal{D}}} % 
\newunicodechar{𝒢}{\ensuremath{\mathcal{G}}}
\newunicodechar{𝒦}{\ensuremath{\mathcal{K}}} % 
\newunicodechar{𝔸}{\ensuremath{\mathds{A}}} % 
\newunicodechar{𝔹}{\ensuremath{\mathds{B}}} % 
\newunicodechar{𝟙}{\ensuremath{\mathds{1}}} % 
\newunicodechar{̷}{\not} % 

\usepackage{natbib}
\usepackage[utf8]{inputenc}
\usepackage{amsmath}
\usepackage{amsthm}
\usepackage{booktabs}
\usepackage{xcolor}
\urlstyle{same}
\usepackage{makecell}
\usepackage{multirow}
\usepackage{rotating}
\usepackage{tabulary}
\usepackage{enumitem}
\newlist{lingex}{enumerate}{3} % easy numbering of examples
\setlist[lingex,1]{parsep=0pt,itemsep=1pt,label=(\arabic*),resume=lingexcount}
\newcommand\onelingex[1]{\begin{lingex}\item #1 \end{lingex}}
\usepackage{mathtools}
\newcommand{\ttr}[1]{\left[\begin{array}{lcl}#1\end{array}\right]}
\newcommand{\tf}[2]{\mathrm{#1} & : & \mathit{#2}\\}
\newcommand{\rf}[2]{\mathrm{#1} & = & \mathit{#2}\\}
\newcommand{\mf}[3]{\mathrm{#1=#2} & : & \mathit{#3}\\}
\newcommand{\type}[1]{$\mathit{#1}$}
\newcommand{\jg}[1]{\noindent \textcolor{blue}{\textbf{\emph{[jg:  #1]}}}}
\usepackage{tikz}
\usetikzlibrary{shapes,arrows,positioning,fit}
\tikzstyle{block} = [draw, rectangle, minimum height=3em, minimum width=3em]
\tikzstyle{virtual} = [coordinate]
\usepackage{wasysym}
\date{\today}
\title{}
\begin{document}


\section{Dialogue manager architecture}
\label{sec:orge902f13}
\label{sec:dm}

We believe that it is crucial to use formal tools which are most
appropriate for the task: one should be able to express the rules of
various genres of dialogue in a concise way, free, to any possible
extent, of irrelevant technical details.  In the view of
\citet{dixon2009plans} this is best done by representing the
information-state of the agents as updatable sets of
propositions. Very often, dialogue-managument rules update subsets
(propositions) of the information state independently from the rest. A
suitable and flexible way to represent such updates are as function
types in linear logic. The domain of the function is the subset of
propositions to update, and the co-domain is the (new) set of
propositions which it replaces.

By using well-known techniques which correspond well with the
intuition of information-state based dialogue management, we are able
to provide a fully working prototype of the components of our
framework:

\begin{enumerate}
\item a proof-search engine based on linear logic, modified to support
inputs from external systems (representing inputs and outputs of
the agent)

\item a set of rules which function as a core framework for dialogue
management (in the style of KoS \citep{ginzburg2012interactive})

\item several examples which use the above to construct potential
applications of the system.
\end{enumerate}
\subsection{Linear rules and proof search}
\label{sec:org5492034}
Typically, and in particular in the archetypal logic programming
language prolog \citep{bratko2001prolog}, axioms and rules are expressed
within the general framework of first order logic. However, several
authors \citep{dixon2009plans,martens2015programming} have proposed
using linear logic \citep{girard1995linear} instead. For our purpose,
the crucial feature of linear logic is that hypotheses may be used
\emph{only once}.

In general, the linear arrow corresponds to \emph{destructive state
updates}. Thus, the hypotheses available for proof search correspond
to the \emph{state} of the system. In our application they will correspond
to the \emph{information state} of the dialogue participant.

In linear logic, normally firing a linear rule corresponds to
triggering an \emph{action} of an agent, and a complete proof corresponds to
a \emph{scenario}, i.e. a sequence of actions, possibly involving action from
several agents.  However, the information state (typically in the
literature and in this paper as well), corresponds to the state of a
\emph{single} agent. Thus, a scenario is conceived as a sequence of actions
and updates of the information state of a single agent \(a\), even
though such actions can be attributed to any other dialogue
participant \(b\). (That is, they are \(a\)'s representation of actions of
\(b\).)  Scenarios can be realised as a sequence of actual actions and
updates. That is, an action can result in sending a message to the
outside world (in the form of speech, movement, etc.). Conversely,
events happening in the outside world can result in extra-logical
updates of the information state (through a model of the perceptory
subsystem).

In our work we are employing an information-state update (ISU)
approach, following several authors, including \citet{larsson2002issue}
and \citet{ginzburg2012interactive}. We treat the information state as a
multiset of \emph{linear hypotheses} that can be queried. Because they are
linear, these hypotheses can also be removed from the state.  In
particular, we have a fixed set of rules (they remain available even
after being used). Each such rule manipulates a part of the
information state (captured by its premisses) and leaves everything
else in the state alone.

\subsection{Questions and answers}
\label{sec:org5c9f5c8}
\label{sec:qa} 

The essential components of the representation of a
question are a type \ensuremath{\Conid{A}}, and a predicate \ensuremath{\Conid{P}} over \ensuremath{\Conid{A}}. Using
a typed intuitionistic logic, we write:

\begin{tabular}{cccc}
   & \ensuremath{\Conid{A}\mathbin{:}\Conid{Type}}   & \quad \quad\quad \quad \quad    &                    \ensuremath{\Conid{P}\mathbin{:}\Conid{A}\to \Conid{Prop}}
\end{tabular}

The intent of the question is to find out about a value \ensuremath{\Varid{x}} of
type \ensuremath{\Conid{A}} which makes \ensuremath{\Conid{P}\;\Varid{x}} true, or at least entertained by the other
participant. 

We make use of metavariables to represent what is being asked, as the
unknown in a proposition.  Within the state of the agent, if the value
of the requested answer is represented as a metavariable \ensuremath{\Varid{x}}, then the
question can be represented as: \ensuremath{\Conid{Q}\;\Conid{A}\;\Varid{x}\;(\Conid{P}\;\Varid{x})}.  That is, the pending
question (\ensuremath{\Conid{Q}} denotes a question constructor) is a triple of a type, a
metavariable \ensuremath{\Varid{x}}, and a proposition where \ensuremath{\Varid{x}} occurs. We stress
that \ensuremath{\Conid{P}\;\Varid{x}} is \emph{not} part of the information state of the agent yet,
rather the fact that the above question is \emph{under discussion} is a
fact. For example, after asking ``Where does John live?'', we have a
fact (we use the double colon for information-state facts and we
assume that agent's questions under discussion are stacked in
the \ensuremath{\Conid{Cons}}-list):

\begin{hscode}\SaveRestoreHook
\column{B}{@{}>{\hspre}l<{\hspost}@{}}%
\column{12}{@{}>{\hspre}l<{\hspost}@{}}%
\column{13}{@{}>{\hspre}l<{\hspost}@{}}%
\column{17}{@{}>{\hspre}l<{\hspost}@{}}%
\column{E}{@{}>{\hspre}l<{\hspost}@{}}%
\>[B]{}\_\mathbin{::}\Conid{QUD}\;{}\<[12]%
\>[12]{}(\Conid{Cons}{}\<[E]%
\\
\>[12]{}\hsindent{1}{}\<[13]%
\>[13]{}(\Conid{Q}\;{}\<[17]%
\>[17]{}\Conid{Location}\;\Varid{x}\;(\Conid{Live}\;\Conid{John}\;\Varid{x}))\;{}\<[E]%
\\
\>[17]{}\Varid{qs}){}\<[E]%
\ColumnHook
\end{hscode}\resethooks

Resolving a question can be done by communicating an answer. An answer
to a question \ensuremath{(\Conid{A}\mathbin{:}\Conid{Type};\Conid{P}\mathbin{:}\Conid{A}\to \Conid{Prop})} can be of either of the two
following forms: i) A \textbf{ShortAnswer}, which is a pair of an element \ensuremath{\Conid{X}\mathbin{:}\Conid{A}} and
its type \ensuremath{\Conid{A}}, represented as \ensuremath{\Conid{ShortAnswer}\;\Conid{A}\;\Conid{X}} or ii) An \textbf{Assertion} which is
a proposition \ensuremath{\Conid{R}\mathbin{:}\Conid{Prop}}, represented as \ensuremath{\Conid{Assert}\;\Conid{R}}.

For more detailed description we refer our reader to the original
paper \citep{maraev2020tal}.

\subsection{Dialogue management}
\label{sec:org8989cb7}
\label{sec:dm-rules} 

Our DM models the information-state of only one
participant. Regardless, this participant can record its own beliefs
about the state of other participants.In general, the core of DM is
comprised of a set of linear-logic rules which depend on the domain of
application. However, many rules will be domain-independent (such as
generic processing of answers). Here we will provide a few examples of
the rules which are implemented in our system, and we refer our reader
to \citep{maraev2020tal} for more detailed description.

The following rule accounts for pushing the content (a question) of
any received \ensuremath{\Conid{Ask}} move on top of the stack of questions under
discussion (\ensuremath{\Conid{QUD}}).

\begin{hscode}\SaveRestoreHook
\column{B}{@{}>{\hspre}l<{\hspost}@{}}%
\column{3}{@{}>{\hspre}l<{\hspost}@{}}%
\column{13}{@{}>{\hspre}l<{\hspost}@{}}%
\column{E}{@{}>{\hspre}l<{\hspost}@{}}%
\>[B]{}\Varid{pushQUD}\mathbin{:}{}\<[13]%
\>[13]{}(\Varid{x}\;\Varid{y}\mathbin{:}\Conid{DP})\to {}\<[E]%
\\
\>[B]{}\hsindent{3}{}\<[3]%
\>[3]{}(\Varid{q}\mathbin{:}\Conid{Question})\to {}\<[E]%
\\
\>[B]{}\hsindent{3}{}\<[3]%
\>[3]{}(\Varid{qs}\mathbin{:}\Conid{List}\;\Conid{Question})\to {}\<[E]%
\\
\>[B]{}\hsindent{3}{}\<[3]%
\>[3]{}\Conid{Pending}\;(\Conid{Ask}\;\Varid{q}\;\Varid{x}\;\Varid{y})\mathbin{⊸}{}\<[E]%
\\
\>[B]{}\hsindent{3}{}\<[3]%
\>[3]{}\Conid{QUD}\;\Varid{qs}\mathbin{⊸}\Conid{QUD}\;(\Conid{Cons}\;\Varid{q}\;\Varid{qs}){}\<[E]%
\ColumnHook
\end{hscode}\resethooks

After the question has been integrated, if the system has a fact \ensuremath{\Varid{p}}
in its database it can produce an answer.\footnote{or, possibly, a domain-specific
clarification request depending on whether the fact is unique and
concrete, see \citep{maraev2020tal} for further details}
\begin{hscode}\SaveRestoreHook
\column{B}{@{}>{\hspre}l<{\hspost}@{}}%
\column{4}{@{}>{\hspre}l<{\hspost}@{}}%
\column{7}{@{}>{\hspre}l<{\hspost}@{}}%
\column{20}{@{}>{\hspre}l<{\hspost}@{}}%
\column{22}{@{}>{\hspre}l<{\hspost}@{}}%
\column{23}{@{}>{\hspre}l<{\hspost}@{}}%
\column{26}{@{}>{\hspre}c<{\hspost}@{}}%
\column{26E}{@{}l@{}}%
\column{34}{@{}>{\hspre}c<{\hspost}@{}}%
\column{34E}{@{}l@{}}%
\column{E}{@{}>{\hspre}l<{\hspost}@{}}%
\>[B]{}\Varid{produceAnswer}\mathbin{:}{}\<[E]%
\\
\>[B]{}\hsindent{4}{}\<[4]%
\>[4]{}(\Varid{a}\mathbin{:}\Conid{Type})\to {}\<[20]%
\>[20]{}(\Varid{x}\mathbin{:}\Varid{a})\to (\Varid{p}\mathbin{:}\Conid{Prop})\to {}\<[E]%
\\
\>[B]{}\hsindent{4}{}\<[4]%
\>[4]{}(\Varid{qs}\mathbin{:}\Conid{List}\;\Conid{Question}){}\<[26]%
\>[26]{}\to {}\<[26E]%
\\
\>[B]{}\hsindent{4}{}\<[4]%
\>[4]{}\Conid{QUD}\;(\Conid{Cons}\;(\Conid{Q}\;\Conid{USER}\;\Varid{a}\;\Varid{x}\;\Varid{p})\;\Varid{qs}){}\<[34]%
\>[34]{}\mathbin{⊸}{}\<[34E]%
\\
\>[B]{}\hsindent{4}{}\<[4]%
\>[4]{}\Varid{p}{}\<[7]%
\>[7]{}\rightarrowtriangle{}\<[E]%
\\
\>[B]{}\hsindent{4}{}\<[4]%
\>[4]{}[\mskip1.5mu {}\<[7]%
\>[7]{}\_\mathbin{::}\Conid{Agenda}\;({}\<[23]%
\>[23]{}\Conid{ShortAnswer}{}\<[E]%
\\
\>[7]{}\hsindent{15}{}\<[22]%
\>[22]{}\Varid{a}\;\Varid{x}\;\Conid{SYSTEM}\;\Conid{USER});{}\<[E]%
\\
\>[7]{}\_\mathbin{::}\Conid{QUD}\;\Varid{qs};{}\<[E]%
\\
\>[7]{}\_\mathbin{::}\Conid{Answered}\;(\Conid{Q}\;\Conid{USER}\;\Varid{a}\;\Varid{x}\;\Varid{p})\mskip1.5mu]{}\<[E]%
\ColumnHook
\end{hscode}\resethooks
Note: taking a linear argument and producing it again is a common
pattern, which can be spelled out \ensuremath{\Conid{A}\mathbin{⊸}[\mskip1.5mu \_\mathbin{::}\Conid{A};\_\mathbin{::}\Conid{P}\mskip1.5mu]}. It is so
common that from here on we use the syntactic sugar \ensuremath{\Conid{A}\rightarrowtriangle\Conid{P}} for it.
\section{Extending the framework with enthymematic reasoning}
\label{sec:org4ece00e}
\label{sec:dm-enthymeme}

In this section we will describe a rudimentary support for
enthymematic reasonin within the framework described above. It appears
to be easier to discuss the extension of the system with a simple
example which alludes to \citet{breitholtz2020}.

\begin{lingex}
\item\label{ex:bypass} 
\begin{tabulary}{\linewidth}{lll}
1 & U: & How can I get home?\\
2 & S: & Via the bypass.\\
3 & U: & Why the bypass?\\
4 & S: & Because the route is the shortest.\\
\end{tabulary}
\end{lingex}

For dealing with lines 1--2 of example \ref{ex:bypass}, let's assume
that system has an access to the following facts from the knowledge
base which represent three possible routes to home, via three
different roads.
\begin{hscode}\SaveRestoreHook
\column{B}{@{}>{\hspre}l<{\hspost}@{}}%
\column{E}{@{}>{\hspre}l<{\hspost}@{}}%
\>[B]{}\_\mathbin{::}\Conid{Route}\;\Conid{Bypass}\;\Conid{Home};{}\<[E]%
\\
\>[B]{}\_\mathbin{::}\Conid{Route}\;\Conid{ParkLane}\;\Conid{Home};{}\<[E]%
\\
\>[B]{}\_\mathbin{::}\Conid{Route}\;\Conid{BridgeRoute}\;\Conid{Home};{}\<[E]%
\ColumnHook
\end{hscode}\resethooks
Assuming that the question under discussion is \ensuremath{(\Conid{Q}\;\Conid{USER}\;\Conid{Road}\;\Varid{x}\;(\Conid{Route}\;\Varid{x}\;\Conid{Home}))}, and the choice of hypothesis is pseudo-random, we can see
that \ensuremath{\Varid{x}} unifies with any of three fact, therefore using
the \ensuremath{\Varid{produceAnswer}} rule, system can produce a short
answer \ensuremath{(\Conid{ShortAnswer}\;\Conid{Road}\;\Conid{Bypass}\;\Conid{SYSTEM}\;\Conid{USER})}, which can be realised
as ``Via the bypass''.

Now let us turn to argumentative part of the dialogue. We would need
to modify a domain specific representation of a question a little bit,
adding additional predicate: \ensuremath{(\Conid{Q}\;\Conid{USER}\;\Conid{Road}\;\Varid{x}\;(\Conid{Pick}\;(\Conid{Route}\;\Varid{x}\;\Conid{Home})))}. Also, knowledge base can be extended with some additional
facts about the qualities of the routes.
\begin{hscode}\SaveRestoreHook
\column{B}{@{}>{\hspre}l<{\hspost}@{}}%
\column{E}{@{}>{\hspre}l<{\hspost}@{}}%
\>[B]{}\_\mathbin{::}\Conid{Shortest}\;(\Conid{Route}\;\Conid{Bypass}\;\Conid{Home});{}\<[E]%
\\
\>[B]{}\_\mathbin{::}\Conid{Cheapest}\;(\Conid{Route}\;\Conid{ParkLane}\;\Conid{Home});{}\<[E]%
\\
\>[B]{}\_\mathbin{::}\Conid{Prettiest}\;(\Conid{Route}\;\Conid{BridgeRoute}\;\Conid{Home});{}\<[E]%
\ColumnHook
\end{hscode}\resethooks

To represent enthymematic reasoning (lines 3--4 of \ref{ex:bypass}), we
will introduce the reasoning pattern represented by following rule:
\begin{hscode}\SaveRestoreHook
\column{B}{@{}>{\hspre}l<{\hspost}@{}}%
\column{4}{@{}>{\hspre}l<{\hspost}@{}}%
\column{6}{@{}>{\hspre}c<{\hspost}@{}}%
\column{6E}{@{}l@{}}%
\column{9}{@{}>{\hspre}l<{\hspost}@{}}%
\column{26}{@{}>{\hspre}c<{\hspost}@{}}%
\column{26E}{@{}l@{}}%
\column{E}{@{}>{\hspre}l<{\hspost}@{}}%
\>[B]{}\Varid{toposShortest}\mathbin{:}(\Varid{x}\mathbin{:}\Conid{Road})\to (\Varid{y}\mathbin{:}\Conid{To})\to {}\<[E]%
\\
\>[B]{}\hsindent{4}{}\<[4]%
\>[4]{}(\Varid{qs}\mathbin{:}\Conid{List}\;\Conid{Question})\to {}\<[E]%
\\
\>[B]{}\hsindent{4}{}\<[4]%
\>[4]{}\Conid{QUD}\;(\Conid{Cons}\;(\Conid{Q}\;\Conid{USER}\;\Conid{Road}\;\Varid{x}{}\<[E]%
\\
\>[4]{}\hsindent{5}{}\<[9]%
\>[9]{}(\Conid{Pick}\;(\Conid{Route}\;\Varid{x}\;\Varid{y})))\;\Varid{qs})\rightarrowtriangle{}\<[E]%
\\
\>[B]{}\hsindent{4}{}\<[4]%
\>[4]{}\Conid{Route}\;\Varid{x}\;\Varid{y}\mathbin{⊸}{}\<[E]%
\\
\>[B]{}\hsindent{4}{}\<[4]%
\>[4]{}\Conid{Shortest}\;(\Conid{Route}\;\Varid{x}\;\Varid{y}){}\<[26]%
\>[26]{}\rightarrowtriangle{}\<[26E]%
\\
\>[4]{}\hsindent{2}{}\<[6]%
\>[6]{}[\mskip1.5mu {}\<[6E]%
\>[9]{}\_\mathbin{::}\Conid{Pick}\;(\Conid{Route}\;\Varid{x}\;\Varid{y});{}\<[E]%
\\
\>[9]{}\_\mathbin{::}\Conid{Topos}\;(\Conid{Shortest}\;(\Conid{Route}\;\Varid{x}\;\Varid{y}));\mskip1.5mu];{}\<[E]%
\ColumnHook
\end{hscode}\resethooks
This can be read as follows: ``In the context of a question under
discussion, involving picking a route, pick the shortest one, and
remember why it was picked''. The latter is represented in the last
line and alludes to the salient \emph{topos} used for this choice. Note, that
here we leave destination underspecified, and further
underspecifications are possible: not only shortest routes might be
preferred but also shortest times or sentences.

Following \citet{breitholtz2020} we treat why-questions as questions
asking for a topos, which becomes the question under discussion \ensuremath{(\Conid{Q}\;\Conid{USER}\;\Conid{Reason}\;\Varid{t}\;(\Conid{Topos}\;\Varid{t}))} where \ensuremath{\Varid{t}} is a metavariable representing the
reason for choosing the bypass. With a local topos produced
by \ensuremath{\Varid{toposShortest}} rule at hand we can apply the
standard \ensuremath{\Varid{produceAnswer}} rule, which would elicit a short answer:
\begin{hscode}\SaveRestoreHook
\column{B}{@{}>{\hspre}l<{\hspost}@{}}%
\column{4}{@{}>{\hspre}l<{\hspost}@{}}%
\column{6}{@{}>{\hspre}l<{\hspost}@{}}%
\column{E}{@{}>{\hspre}l<{\hspost}@{}}%
\>[B]{}\Conid{ShortAnswer}\;\Conid{Reason}\;{}\<[E]%
\\
\>[B]{}\hsindent{4}{}\<[4]%
\>[4]{}(\Conid{Shortest}\;(\Conid{Route}\;\Conid{Bypass}\;\Conid{Home}))\;{}\<[E]%
\\
\>[4]{}\hsindent{2}{}\<[6]%
\>[6]{}\Conid{SYSTEM}\;\Conid{USER}{}\<[E]%
\ColumnHook
\end{hscode}\resethooks
It can be realised as an utterance ``Because it is shorter'',
concluding our example \ref{ex:bypass}.

Our system allows having several competing topoi, for instance we can
as well analogously add \ensuremath{\Varid{toposPrettiest}} and \ensuremath{\Varid{toposCheapest}}
rules. Assuming pseudo-random selection of an applicable rule system
will pick a pseudo-random route, but still be able to justify such
decision.

\bibliography{../LACATODA2021/lacatoda}
\bibliographystyle{named}
\end{document}
